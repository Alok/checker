\documentclass[11pt]{article}
\usepackage{graphicx}
\usepackage{amscd, amssymb}
\usepackage{enumerate}
\usepackage{mathtools}          %to get \vcentcolon
\usepackage{mathpartir}
\input xy \xyoption{all} \CompileMatrices

\textwidth = 6.5 in
\textheight = 9 in
\oddsidemargin = 0.0 in
\evensidemargin = 0.0 in
\topmargin = 0.0 in
\headheight = 0.0 in
\headsep = 0.0 in
\parindent = 0.0in

\renewcommand{\thesubsection}{\arabic{subsection}}

\newenvironment{eq}{\begin{equation}}{\end{equation}}

\newenvironment{proof}{{\bf Proof}:}{\vskip 5mm }
\newenvironment{rem}{{\bf Remark}:}{\vskip 5mm }
\newenvironment{remarks}{{\bf Remarks}:\begin{enumerate}}{\end{enumerate}}
\newenvironment{examples}{{\bf Examples}:\begin{enumerate}}{\end{enumerate}}  

\newtheorem{proposition}{Proposition}[subsection]
\newtheorem{lemma}[proposition]{Lemma}
\newtheorem{definition}[proposition]{Definition}
\newtheorem{theorem}[proposition]{Theorem}
\newtheorem{cor}[proposition]{Corollary}
\newtheorem{conjecture}{Conjecture}
\newtheorem{pretheorem}[proposition]{Pretheorem}
\newtheorem{hypothesis}[proposition]{Hypothesis}
\newtheorem{example}[proposition]{Example}
\newtheorem{remark}[proposition]{Remark}
\newtheorem{ex}[proposition]{Exercise}
\newtheorem{cond}[proposition]{Conditions}
\newtheorem{cons}[proposition]{Construction}

%\newcommand{\C}[4]{\left[\begin{array}{rcl}&#1\\#3&\dw\dw&#4\\&#2\end{array}\right]}
\newcommand{\llabel}[1]{\label{#1}[{\bf #1}]}
%\newcommand{\llabel}[1]{\label{#1}}
\newcommand{\comment}[1]{}
\newcommand{\sr}{\rightarrow}
\newcommand{\lr}{\longrightarrow}
\newcommand{\xr}{\xrightarrow}
\newcommand{\dw}{\downarrow}
\newcommand{\bdl}{\bar{\Delta}}
\newcommand{\zz}{{\bf Z\rm}}
\newcommand{\zq}{{\bf Z}_{qfh}}
\newcommand{\nn}{{\bf N\rm}}
\newcommand{\qq}{{\bf Q\rm}}
\newcommand{\nq}{{\bf N}_{qfh}}
\newcommand{\oo}{\otimes}
\newcommand{\uu}{\underline}
\newcommand{\ih}{\uu{Hom}}
\newcommand{\af}{{\bf A}^1}
\newcommand{\wt}{\widetilde}
\newcommand{\gm}{{\bf G}_m}
\newcommand{\dsr}{\stackrel{\sr}{\scriptstyle\sr}}
%\newcommand{\PP}{$P_{\infty}$}
\newcommand{\tp}{\tilde{D}}
\newcommand{\HH}{$H_{\infty}$}
\newcommand{\ii}{\stackrel{\scriptstyle\sim}{\sr}}
\newcommand{\BB}{_{\bullet}}
\newcommand{\D}{\Delta}
\newcommand{\colim}{{\rm co}\hspace{-1mm}\lim}
\newcommand{\cf}{{\it cf} }
\newcommand{\msf}{\mathsf }
\newcommand{\mcal}{\mathcal }
\newcommand{\ep}{\epsilon}
\newcommand{\tl}{\widetilde}
\newcommand{\ub}{\mbox{\rotatebox{90}{$\in$}}}
\newcommand{\ssp}{\,\,\,\,\,\,\,\,}
\newcommand{\red}{\twoheadrightarrow}
\newcommand{\eqg}{\stackrel{\Gamma}{\approx}}
\newcommand{\alphaeq}{\stackrel{\alpha}{\sim}}
\newcommand{\rtr}{\triangleright}

\newcommand{\piece}{\vskip 3mm\noindent\refstepcounter{proposition}{\bf
\theproposition}\hspace{2mm}}
\newcommand{\subpiece}{\vskip 3mm\noindent\refstepcounter{equation}{\bf\theequation}
\hspace{2mm}}{\vskip
3mm}

\newcommand{\TT}{{\bf Type0}}
\newcommand{\PP}{{\bf Prop}}
\newcommand{\cc}{{\bf c}}
\newcommand{\rect}{{\bf rect}}
\newcommand{\syndef}{\equiv}
\newcommand{\eqdef}{=}

\newcommand{\cu}{{u}}
\newcommand{\Univ}{{\cal U}}
\newcommand{\univ}{\upsilon}
\newcommand{\funiv}{\upsilon^F}
\newcommand{\FUniv}{{\cal U}^F}
\newcommand{\JJ}[2]{{J(#1,#2)\,}}
\newcommand{\J}[1]{{J(#1)\,}}
\newcommand{\jj}[3]{{j(#1,#2,#3)\,}}
\newcommand{\jr}[2]{{jr(#1, #2)\,}}
\newcommand{\jl}[2]{{jl(#1, #2)\,}}
\newcommand{\UU}[1]{{UU(#1)}}
\newcommand{\rd}{\succ}
\newcommand{\rde}{\succeq}
\newcommand{\rdA}{\rd_{\AA}}
\newcommand{\brd}{\equiv}
\newcommand{\ind}{Ind\,\,}
\newcommand{\erd}[1]{\equiv_{{\AA},#1}}
\renewcommand{\AA}{{\cal A}}

\newcommand{\hastype}{\vcentcolon}

\newcommand{\TYPE}{\mathop{\text{ \sc type}}}
\newcommand{\FTYPE}{\mathop{\text{ \sc type}_F}}
\newcommand{\Type}{\mathop{\text{Type}}}
\newcommand{\Okay}{\mathop{\text{ \sc okay}}}
\newcommand{\Context}{\vdash\Okay}
\renewcommand{\Context}{\vdash}
\newcommand{\ha}[2]{#1[#2]}
\newcommand{\Wrefl}{{\sf Wrefl}}
\newcommand{\weta}{{\sf weta}}
\newcommand{\Wtrans}{{\sf Wtrans}}
\newcommand{\Wsymm}{{\sf Wsymm}}
\newcommand{\El}{{\sf El}}
\newcommand{\tprod}{\Pi}
\newcommand{\annot}{{\sf annot}}
\newcommand{\haa}[2]{\ha\annot{#1,#2}}
\renewcommand{\haa}[2]{#1}
\newcommand{\conv}{{\sf conv}}
\newcommand{\wconv}{{\sf wconv}}
\newcommand{\ev}{{\sf ev}}
\newcommand{\wbeta}{{\sf wbeta}}
\newcommand{\weleq}{{\sf weleq}}
\newcommand{\welfeq}{{\sf welfeq}}
\newcommand{\weveq}{{\sf wev}}
\newcommand{\wlambda}{{\sf wlambda}}
\newcommand{\Wsubst}{{\sf Wsubst}}
\newcommand{\wsubst}{{\sf wsubst}}
\newcommand{\wpieq}{{\sf wpi}}
\newcommand{\uep}{{\sf uep}}
\newcommand{\wrefl}{{\sf wrefl}}
\newcommand{\wsymm}{{\sf wsymm}}
\newcommand{\wtrans}{{\sf wtrans}}
\newcommand{\windo}{{\sf windo}}
\newcommand{\windeq}{{\sf windeq}}
\newcommand{\winds}{{\sf winds}}
\newcommand{\wetaN}{{\sf wetaN}}
\newcommand{\wetatN}{{\sf wetatN}}
\newcommand{\succN}{{\sf succ}}
\newcommand{\zeroN}{{\sf zero}}
\newcommand{\var}{\char`_}
\newcommand{\defn}{\vcentcolon=}
\newcommand{\NN}{\mathbb N}
%\newcommand{\UNIV}{\mathop{\text{ \sc univ}}}
\newcommand{\UNIV}{\hastype \NN}
\newcommand{\indN}{{\sf ind}_\NN}
\newcommand{\indtN}{{\sf indt}_\NN}

\begin{document}

\parskip = 2mm
\begin{center}
{\bf\Large A test type system}

{\bf Vladimir Voevodsky}

{Started January 25, 2013}  

{This version modified heavily by Dan Grayson}  
\end{center}

\tableofcontents

This note gives some ideas about the test type system TTS with secondary
witnessed which Dan Grayson and I have been working on implementing. While TTS
by itself has (most likely) decidable definitional equality and typing making
secondary witnesses to be formally speaking unnecessary, they become essential
for the implementation of more complex systems with undecidable typing such as
HTS.

We are aiming at a type system where every derivable extended sentence can be
obtained by a unique inference rules such that one gets a bijection between
inference trees and $\alpha$-equivalence classes of derivable extended
sentences.  The extension is chosen here in a minimal way.

\subsection{Rules}

A {\em context entry} $J$ has one of the following forms:

$$ x \hastype T $$
$$ x \defn t \hastype T $$
$$ X \TYPE $$
$$ X \defn T \TYPE $$
$$ X \FTYPE $$
$$ X \defn T \FTYPE $$

where $x$ or $X$ is the variable {\em introduced} by the context entry, and
where $t$ and $T$ are expressions, possibly with free variables, that {\em
  define} values for the variables (as indicated by the operator $\defn$).  The
form $\FTYPE$ is used to distinguish types that are {\em fibrant}.

A {\em context} or {\em sentence} $\Gamma$ is a list of context entries, each of
which has all of its free variables bound by variables introduced earlier in
the list.  Alternatively, we may say the variable introduced by a context entry in
the list is bound in the remainder of the list.

Inference rules given later determine which sentences are well-formed and
derivable.  The notation $$\Gamma \Context$$ will be used to indicate that
$\Gamma$ is well-formed.

Syntactic equality between expressions occurring twice in the conclusion of an
inference rule will be up to renaming of local variables, expansion of
definitions, and a few other rules, to be presented later, that amount to
ignoring certain subexpressions during equality checking.  We use the notation
$ t \equiv t' $ for syntactic equality.

The last variable introduced by a context is always unused, so we may 
abbreviate the sentence
$$ \Gamma, x \defn t \hastype T \Context $$ by
$$ \Gamma \vdash t \hastype T $$
and
$$ \Gamma, X \defn T \TYPE \Context $$ by
$$ \Gamma \vdash T \TYPE $$
and
$$ \Gamma, X \defn T \FTYPE \Context $$ by
$$ \Gamma \vdash T \FTYPE. $$

The abbreviation emphasizes that the forms without defined values serve double
duty, either for introducing a variable of type $T$ into the context to serve
as a hypothesis, or as a way of asserting that $T$ is a type.  The hypotheses
{\em active} in a context $\Gamma$ are the variables with no defined value.

Variables introduced by context entries in the context may have names that are
uninteresting for us, if they aren't referred to again.  To indicate that, we
will write the names of such variables as underscores, as in $\var
\hastype T_1,\dots,\var \hastype T_n\Context$.

Putting defined values explicitly into the context allows us to identify a map
$$ [ x_1 : T_1 , \dots , x_m : T_m ] \xrightarrow u [ y_1 : U_1 , \dots , y_n : U_n ] $$
in the contextual category with its graph 
$$ [ x_1 : T_1 , \dots , x_m : T_m, y_1 \defn u_1 : U_1 , \dots , y_n \defn u_n
  : U_n ],$$ provided we regard variables with defined values as mere annotations, rather than as coordinates
with specified values.  That is convenient, because of its compact form, without multiple
explicit substitutions.  Moreover, a canonical pullback square has the form
\[\xymatrix{
[ x_1 : T_1 , \dots , x_m : T_m, y_1 \defn u_1 : U_1 , \dots , y_n \defn u_n : U_n, z : V ] \ar[r] \ar[d] &
[ y_1 : U_1 , \dots , y_n : U_n, z:V ] \ar[d] \\
[ x_1 : T_1 , \dots , x_m : T_m ] \ar[r]^u &
[ y_1 : U_1 , \dots , y_n : U_n ]
}.\]
Introducing a type that serves as definitional equality and can thus be placed in the context, as we will do, will allow us
to produce arbitrary pullbacks in the contextual category.

Thanks to Dan Licata and Peter Lumsdaine for suggestions and advice.

\paragraph{Inference rules} 

\begin{enumerate}

\item
\[\inferrule{\ }{\Context}\]

\item
The effect of defined values:
\[\inferrule{\Gamma,x \defn t \hastype T,\Gamma'\Context}{x \syndef t}\]

\item
\[\inferrule{\Gamma,x:Q,\Gamma'\Context}{\Gamma,x:Q,\Gamma',y\defn x : Q \Context}\]

\[\inferrule{\Gamma,x \defn p :Q,\Gamma'\Context}{\Gamma,x \defn p :Q,\Gamma',y\defn x : Q \Context}\]

Here (and below) $Q$ represents the tail end of a context entry, not including
the variable or its defined value.  For example, a special case of this rule is the following one.

\[\inferrule{\Gamma,x:T,\Gamma'\Context}{\Gamma,x:T,\Gamma' \vdash x : T}\]

\item
\[\inferrule{\Gamma, x:Q,\Gamma' \Context \\ \Gamma, x \defn t:Q \Context}{\Gamma, x \defn t:Q,\Gamma'\Context}\]

For example, special cases of this rule are the following ones.

\[\inferrule{\Gamma, X \TYPE,\Gamma' \Context \\ \Gamma \vdash T \Type}{\Gamma, X \defn T \TYPE, \Gamma'\Context}\]

\[\inferrule{\Gamma, x:T,\Gamma' \Context \\ \Gamma \vdash t:T}{\Gamma, x \defn t:T,\Gamma'\Context}\]

\item
If the variable $x$ does not appear in $\Gamma'$, then

\[\inferrule{\Gamma,x\hastype Q,\Gamma'\Context}{\Gamma,\Gamma' \Context }\]

and

\[\inferrule{\Gamma, x\defn t\hastype Q,\Gamma'\Context}{\Gamma,\Gamma' \Context }\]

In the latter rule, observe that since we are working modulo unfolding of
definitions, $\Gamma'$ can always be replaced by something that doesn't involve
$x$; that amounts to substituting $t$ for $x$ in $\Gamma'$.

\item
\[\inferrule{
  \Gamma \vdash T \TYPE  
  \\
  \Gamma \vdash T' \TYPE }
{\Gamma \vdash T \eqdef T' \TYPE}\]

This inference rule introduces a type called $T \eqdef T'$.  It will serve as
the mathematical notion of exact equality between types.

\item
  \[\inferrule{
  \Gamma \vdash p \hastype  T \eqdef T'
  \\
  \Gamma \vdash t \hastype T
  \\
  \Gamma \vdash t' \hastype T'
  } {
  \Gamma \vdash t \eqdef t' \hastype T \eqdef T' \TYPE}\]

This inference rule introduces a type called $t \eqdef t' \hastype T \eqdef
T'$.  It will serve as the mathematical notion of exact equality between
objects of equal types.  In the case where $T$ and $T'$ are the same, we will
abbreviate it to $t \eqdef t' \hastype T$.

\item
\[\inferrule
{\Gamma \vdash p \hastype T \eqdef T' \\ \Gamma \vdash p' \hastype T \eqdef T'}
{\Gamma \vdash \ha\uep{} \hastype p \eqdef p' \hastype T \eqdef T'}
\]
Proofs of equality between types are equal.

\[\inferrule
{\Gamma \vdash p \hastype t \eqdef t' \hastype T \eqdef T' \\ \Gamma \vdash p' \hastype t \eqdef t' \hastype T \eqdef T'}
{\Gamma \vdash \ha\uep{} \hastype p \eqdef p' \hastype (t \eqdef t' \hastype T \eqdef T') \eqdef (t \eqdef t' \hastype T \eqdef T')}
\]
Proofs of equality between objects are equal.  We regard the colon as a right
associative operator, so the rule can be written in the following abbreviated form.
\[\inferrule
{\Gamma \vdash p \hastype t \eqdef t' \hastype T \eqdef T' \\ \Gamma \vdash p' \hastype t \eqdef t' \hastype T \eqdef T'}
{\Gamma \vdash \ha\uep{} \hastype p \eqdef p' \hastype t \eqdef t' \hastype T \eqdef T'}
\]

Maybe it is better to implement this at a lower level, by implementing $t \equiv t'$ so that
$ p \equiv p' $ is always true.

\item 
\[\inferrule{\Gamma \vdash T \TYPE \\ \Gamma \vdash T' \TYPE \\ T \syndef T'}
       {\Gamma \vdash \ha\Wrefl{}\hastype  T\eqdef T'}
\]

\item 
\[\inferrule{\Gamma \vdash p\hastype T_1\eqdef T_2 }{\Gamma \vdash \ha\Wsymm{p}\hastype T_2\eqdef T_1}\]

\item 
\[
  \inferrule{
    \Gamma \vdash p\hastype T_1\eqdef T_2
    \\
    \Gamma \vdash q\hastype T_2\eqdef T_3
  }
  {\Gamma \vdash \ha\Wtrans{p,T_2,q}\hastype  T_1\eqdef T_3 }
\]

\item 
\[\inferrule{
  \Gamma \vdash o \hastype{T}
  \\
  \Gamma \vdash o' \hastype{T'}
  \\
  T \syndef T'
  \\
  o \syndef o'
  } {
  \Gamma \vdash  \ha\wrefl{}\hastype  o \eqdef o' \hastype T \eqdef T'
  } 
\]

\item 
\[\inferrule
    {\Gamma \vdash p\hastype o_1\eqdef o_2 \hastype T_1 \eqdef T_2}
    {\Gamma \vdash \ha\wsymm{p}\hastype o_2\eqdef o_1 \hastype T_2 \eqdef T_1}\]

\item 
\[\inferrule{\Gamma \vdash p\hastype o_1\eqdef o_2 \hastype T_1 \eqdef T_2
  \\
  \Gamma \vdash q\hastype o_2\eqdef o_3 \hastype T_2 \eqdef T_3
  } {
  \Gamma \vdash \ha\wtrans{p,o_2,T_2,q}\hastype o_1\eqdef o_3 \hastype T_1 \eqdef T_3}
\]

\item 
Equal objects have equal elements:
\begin{mathparpagebreakable}
\inferrule
    { \Gamma \vdash t \hastype T \\ \Gamma \vdash p \hastype  T \eqdef T'}
    { \Gamma \vdash \ha\conv{t,T,p} \hastype T' }
\and
\inferrule
    { \Gamma \vdash t \hastype T \\ \Gamma \vdash p \hastype  T \eqdef T' }
    { \Gamma \vdash \ha\wconv{} \hastype t \eqdef \ha\conv{t,T,p} \hastype T \eqdef T'}
\end{mathparpagebreakable}

\item
Irrelevance of universe level:
\[\inferrule{ t \syndef t' }{ \ha\El{n,t} \syndef \ha\El{n',t'} }\]

\item
For each $n \hastype  \NN$ we introduce expressions $\univ_n$ and $\funiv_n$.  We introduce $\Univ_n$ as an
abbreviation for $\ha\El{\ha\succN n, \univ_n}$ and $\FUniv_n$ as an
abbreviation for $\ha\El{\ha\succN n, \funiv_n}$.

\begin{mathparpagebreakable}
\inferrule{\Gamma \vdash n \UNIV  } {\Gamma \vdash \univ_n \hastype  \FUniv_{\ha\succN n}  }
\and
\inferrule{\Gamma \vdash n \UNIV  } {\Gamma \vdash \funiv_n \hastype  \FUniv_{\ha\succN n} }
\end{mathparpagebreakable}

\item
\ 
\begin{mathparpagebreakable}
\inferrule{ \Gamma \vdash o \hastype{\Univ_n}}
          { \Gamma \vdash \ha\El{n,o}\TYPE } \and
\inferrule{ \Gamma \vdash o \hastype{\FUniv_n}}
          { \Gamma \vdash \ha\El{n,o}\FTYPE } \and
\inferrule{ \Gamma \vdash T \FTYPE }
          { \Gamma \vdash T \TYPE } \and
\inferrule{ \Gamma \vdash o \hastype{\FUniv_n} }
          { \Gamma \vdash o \hastype{\Univ_n} } \and
\inferrule{\Gamma \vdash p \hastype  o \eqdef o' \hastype \Univ_n \eqdef \Univ_{n'}} 
          {\Gamma \vdash \ha\weleq{p} \hastype  \ha\El{n,o} \eqdef \ha\El{n',o'}}
\end{mathparpagebreakable}


\item 
\[\inferrule{ \Gamma, \var \hastype T \vdash U\TYPE  }{\Gamma \vdash \ha\tprod{T,.U}\TYPE}\]

Here $.U$ is notation for abstraction: binding an anonymous variable to the
expression $U$; we will also use the notation $x.U$ when binding a named
variable $x$, as in the following equivalent version of the same rule.

\[\inferrule{ \Gamma, x \hastype T \vdash U\TYPE  }
       {\Gamma \vdash \ha\tprod{T,x.U}\TYPE}\]

\item 
\[\inferrule{
  \Gamma \vdash T \FTYPE \\
  \Gamma, \var \hastype T \vdash U \FTYPE  
}{
  \Gamma \vdash \ha\tprod{T,.U} \FTYPE
}\]

\item 
\[\inferrule{
  \Gamma, \var \hastype T \vdash u \hastype{U}
}{\Gamma \vdash \ha\lambda{.u} \hastype{\ha\tprod{T,.U}}}
\]

\item 
\[ \inferrule{ 
  \Gamma \vdash p\hastype T \eqdef T'  \\
  \Gamma, x \hastype T \vdash U \TYPE \\
  \Gamma, x' \hastype T' \vdash U' \TYPE \\
  \Gamma, x \hastype T, x' \hastype T', w \hastype  x \eqdef x' \hastype T \eqdef T' \vdash q \hastype  U[x] \eqdef U'[x'] 
  } {
  \Gamma \vdash \ha\wpieq{p,...q}\hastype \ha\tprod{T,x.U}\eqdef \ha\tprod{T',x'.U'}
}\]

\item 
\[ \inferrule{ 
  \Gamma \vdash p\hastype T \eqdef T' \\
  \Gamma, x \hastype T \vdash u \hastype  U \\
  \Gamma, x' \hastype T' \vdash u' \hastype  U' \\
  \Gamma, x \hastype T, x' \hastype T', w \hastype  x \eqdef x' \hastype T \eqdef T' \vdash q \hastype  u[x] \eqdef u'[x'] \hastype  U[x] \eqdef U'[x'] 
  } {
  \Gamma \vdash \ha\wlambda{p,...q} \hastype  \ha\lambda{.u}\eqdef \ha\lambda{.u'} \hastype  \ha\tprod{T,.U} \eqdef \ha\tprod{T',.U'}
}\]

\item 
\[\inferrule{
  \Gamma \vdash f \hastype{\ha\tprod{T,.U}}
  \\ 
  \Gamma \vdash t \hastype{T}
  \\
  U' \syndef U[\haa t T] 
}{
  \Gamma \vdash \ha \ev{f,t,{\ha\tprod{T,.U}}} \hastype U'
}\]

%% The annotation is present to ensure that the sentence $$\Gamma, \var \defn
%% U[\haa t T] \TYPE \Context$$ is checkable.

Question: what makes the sentence 
\[ \Gamma \vdash U[\haa t T] \TYPE \]
checkable, in the context above?

The rule above can be expressed without using substitution, as follows.
\[\inferrule{
  \Gamma \vdash f \hastype{\ha\tprod{T,x.U}}
  \\ 
  \Gamma \vdash t \hastype{T}
}{
  \Gamma, x \defn t \hastype  T \vdash \ha \ev{f,t,{\ha\tprod{T,x.U}}} \hastype U
}\]
The disadvantages are that the conclusion has something other than $\Gamma$ to the
left of the turnstile, and the name of the variable becomes important.

\item 
\[\inferrule{
   \Gamma \vdash p \hastype  f \eqdef f' \hastype {\ha\tprod{T,.U}} \eqdef {\ha\tprod{T',.U'}} 
   \\ 
   \Gamma \vdash q \hastype  t \eqdef t' \hastype T \eqdef T'
   }{
   \Gamma \vdash \ha\weveq{p,q} \hastype  \ha\ev{f,t,{\ha\tprod{T,.U}}} \eqdef \ha\ev{f',t',{\ha\tprod{T',.U'}}} \hastype U[\haa t T] \eqdef U'[\haa {t'} {T'}]
  }\]

\item 
% beta
\[\inferrule{
  \Gamma \vdash \ha\lambda{.u} \hastype {\ha\tprod{T,.U}}
  \\
  \Gamma \vdash t \hastype{T} 
  }{
  \Gamma \vdash \ha\wbeta{} \hastype  \ha\ev{\ha\lambda{.u},t,{\ha\tprod{T,.U}}} \eqdef u[\haa{t}{T}] \hastype  U[\haa{t}{T}]
}\]

%% Replacing $\ha\lambda{.u}$ by $\haa{\ha\lambda{.u}}{\ha\tprod{T,.U}}$ would
%% be an alternative to putting $T$ and $.U$ in the witness.  Comparison of two
%% terms of the same type by normalization need not involve computation of the
%% types of partially applied results, so a version of this rule free of type
%% annotations is needed.

%% (It would be better if $\beta$-reduction were free-running, without the
%% necessity of tracking the types $T$ and $U$.)

% eta
\item 
\[
  \inferrule{\Gamma \vdash f \hastype{\ha\tprod{T,.U}}}
       {\Gamma \vdash \ha\weta{} \hastype  f \eqdef \ha\lambda{t.\ha\ev{f,t,{\ha\tprod{T,.U}}}} \hastype  \ha\tprod{T,.U}
       }
\]

\item\ 
\begin{mathparpagebreakable}
  \inferrule{\Gamma \Context}
       {\Gamma \vdash \NN \TYPE }
  \and
  \inferrule{\Gamma \Context}
       {\Gamma \vdash \zeroN \hastype \NN }
  \and
  \inferrule{ \Gamma \vdash n \hastype \NN }
       {\Gamma \vdash \ha \succN {n} \hastype \NN }
\end{mathparpagebreakable}

\item

This rule and the following three deal with defining a family of types by
induction on $\NN$.  
%% If we treat types as objects of a type called $\Type$,
%% they won't be necessary.

\[
\inferrule{
  \Gamma \vdash B \TYPE
  \\
  \Gamma, n \hastype \NN, C \TYPE \vdash D \TYPE
  \\
  \Gamma \vdash n \hastype \NN
}
     {\Gamma \vdash \ha\indtN{B,..D,n} \TYPE }
\]

\item
\[
\inferrule{
  \Gamma \vdash B \TYPE
  \\
  \Gamma, n \hastype \NN, C \TYPE \vdash D \TYPE
}
     {\Gamma \vdash \ha\windo{} \hastype  \ha\indtN{B,..D,\zeroN} \eqdef B }
\]

\item
\[
\inferrule{
  \Gamma \vdash B \TYPE
  \\
  \Gamma, n \hastype \NN, C \TYPE \vdash D \TYPE
  \\
  \Gamma \vdash n \hastype \NN
}
     {\Gamma \vdash \ha\winds{} \hastype  \ha\indtN{B,..D,\succN[n]} \eqdef D[n,\ha\indtN{B,..D,n}] }
\]

\item
\[
\inferrule{
  \Gamma \vdash q \hastype  B \eqdef B'
  \\
  \Gamma, n \hastype \NN, C \TYPE \vdash D \TYPE
  \\
  \Gamma, n \hastype \NN, C' \TYPE \vdash D' \TYPE
  \\
  \Gamma, n \hastype \NN, \var \hastype  C \eqdef C' \vdash r \hastype  D[n,C] \eqdef D'[n,C']
  \\
  \Gamma \vdash s \hastype  n \eqdef n' \hastype \NN
}
     {\Gamma \vdash \ha\windeq{q,..r,s} \hastype  \ha\indtN{B,..D,n} \eqdef \ha\indtN{B',..D',n'} }
\]

\item
\[
\inferrule{
  \Gamma \vdash b \hastype T[\zeroN]
  \\
  \Gamma, n \hastype \NN, c \hastype T \vdash d \hastype T[\succN[n]]
  \\
  \Gamma \vdash n \hastype \NN
}
     {\Gamma \vdash \ha\indN{.T,b,..d,n} \hastype  T[n] }
\]

\item
\[
\inferrule{
  \Gamma \vdash b \hastype T[\zeroN]
  \\
  \Gamma, n \hastype \NN, c \hastype T \vdash d \hastype T[\succN[n]]
}
     {\Gamma \vdash \ha\windo{} \hastype  \ha\indN{.T,b,..d,\zeroN} \eqdef b \hastype  T[\zeroN] }
\]

\item
\[
\inferrule{
  \Gamma \vdash b \hastype T[\zeroN]
  \\
  \Gamma, n \hastype \NN, c \hastype T \vdash d \hastype T[\succN[n]]
  \\
  \Gamma \vdash n \hastype \NN
}
     {\Gamma \vdash \ha\winds{} \hastype  \ha\indN{.T,b,..d,\succN[n]} \eqdef d[n,\ha\indN{.T,b,..d,n}] \hastype  T[\succN[n]] }
\]

\item
\[
\inferrule{
  \Gamma, n \hastype \NN \vdash p \hastype  T \eqdef T'
  \\
  \Gamma \vdash q \hastype  b \eqdef b' \hastype T[\zeroN] \eqdef T'[\zeroN]
  \\
  \Gamma, n \hastype \NN, c \hastype T[n] \vdash d \hastype T[\succN[n]]
  \\
  \Gamma, n \hastype \NN, c' \hastype T'[n] \vdash d' \hastype T'[\succN[n]]
  \\
  \Gamma, n \hastype \NN, \var \hastype  c \eqdef c' \hastype T[n] \eqdef T'[n] \vdash r \hastype  d[n,c] \eqdef d'[n,c'] \hastype T[\succN[n]] \eqdef T'[\succN[n]]
  \\
  \Gamma \vdash s \hastype  n \eqdef n' \hastype \NN
}
     {\Gamma \vdash \ha\windeq{.p,q,..r,s} \hastype  \ha\indN{.T,b,..d,n} \eqdef \ha\indN{.T',b',..d',n'} \hastype  T[n] \eqdef T'[n'] }
\]

\item
\[
  \inferrule {
    \Gamma \vdash p \hastype  T[\zeroN] \eqdef T'[\zeroN]
    \\
    \Gamma, n \hastype \NN, \var \hastype  T \eqdef T' \vdash q \hastype  T[\succN[n]] \eqdef T'[\succN[n]]
    \\
    \Gamma \vdash n \hastype \NN
  } {
    \Gamma \vdash \ha\wetatN{.T,.T',p,..q,n} : T[n] \eqdef T'[n]
  }
\]

\item
\[
  \inferrule {
    \Gamma, n \hastype \NN \vdash t : T
    \\
    \Gamma, n \hastype \NN \vdash t' : T'
    \\
    \Gamma \vdash p : t[\zeroN] \eqdef t'[\zeroN] : T[\zeroN] \eqdef T'[\zeroN]
    \\
    \Gamma, n \hastype \NN, \var : t[n] \eqdef t'[n] : T[n] \eqdef T'[n] \vdash q : t[\succN[n]] \eqdef t'[\succN[n]] : T[\succN[n]] \eqdef T'[\succN[n]]
    \\
    \Gamma \vdash n \hastype \NN
  } {
    \Gamma \vdash \ha\wetaN{.T,.T',.t,.t',p,..q,n} : t[n] \eqdef t'[n] : T[n] \eqdef T'[n]
  }
\]

\item
The following rules are included, because they correspond to meta-theorems that
we intend should hold; they are not expected to be useful in constructing
proofs, and they shouldn't participate in derivations considered by the
uniqueness statement of the introduction.

\begin{mathparpagebreakable}
  \inferrule{\Gamma \vdash t \hastype T}
       {\Gamma, \var \hastype T \Context}
  \and
  \inferrule{\Gamma, \var \hastype T \Context}
       {\Gamma \vdash T \TYPE}
  \and
  \inferrule{\Gamma \vdash p \hastype T \eqdef T'}
       {\Gamma, \var \hastype T \eqdef T' \Context}
  \and
  \inferrule{\Gamma, \var \hastype T \eqdef T' \Context}
       {\Gamma \vdash T \TYPE}
  \and
  \inferrule{\Gamma, \var \hastype T \eqdef T' \Context}
       {\Gamma \vdash T' \TYPE}
  \and
  \inferrule{\Gamma \vdash p : t \eqdef t' \hastype T \eqdef T'}
       {\Gamma, \var : t \eqdef t' \hastype T \eqdef T' \Context}
  \and
  \inferrule{\Gamma, \var : t \eqdef t' \hastype T \eqdef T' \Context}
       {\Gamma \vdash t \hastype T}
  \and
  \inferrule{\Gamma, \var : t \eqdef t' \hastype T \eqdef T' \Context}
       {\Gamma \vdash t' \hastype T'}

\end{mathparpagebreakable}

\item
It should be a meta-theorem that if
\[{\Gamma \vdash p : t \eqdef t' \hastype T \eqdef T'}\]
is derivable, then, for some $p'$,
\[{\Gamma \vdash p' : T \eqdef T'}\]
is derivable.  Alternatively, it might be a good idea to keep track of the
witness $p'$ by writing something like
\[{\Gamma, \var \defn p : t \eqdef t' \hastype p' : T \eqdef T' \Context}\]
or
\[{\Gamma, \var : t \eqdef t' \hastype p' : T \eqdef T' \Context}\]

\end{enumerate}

\bibliography{alggeom}
\bibliographystyle{plain}
\end{document}
  %% Local Variables:
  %% compile-command: "pdflatex TTS-internal-witnesses.tex "
  %% End:
